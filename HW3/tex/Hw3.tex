
\documentclass[article]{memoir}

% ----- Packages -----
\usepackage{amsmath}
\usepackage{amsfonts}
\usepackage{amssymb}
% Must come after amsmath.
%\usepackage{amsthm}

\usepackage{mathtools}
\usepackage{subdepth}
\usepackage{dsfont}
\usepackage{flafter}
\usepackage{graphicx}
\usepackage{listings}

% ----- Page -----
\setlrmarginsandblock{1in}{1in}{*}
\setulmarginsandblock{1in}{1in}{*}
\checkandfixthelayout

\makepagestyle{name}
    \makeevenhead{name}{\textsl\thetitle}{}
    {\textsl{\thepage\ of \thelastpage\ --- Nick Ulle}}
    \makeoddhead{name}{\textsl\thetitle}{}
    {\textsl{\thepage\ of \thelastpage\ --- Nick Ulle}}
\makepagestyle{no-name}
    \makeevenhead{no-name}{\textsl\thetitle}{}
    {\textsl{\thepage\ of \thelastpage}}
    \makeoddhead{no-name}{\textsl\thetitle}{}
    {\textsl{\thepage\ of \thelastpage}}

% ----- Floats -----
\changecaptionwidth
\captionwidth{0.6\textwidth}
\setfloatadjustment{table}{\centering}
\setfloatadjustment{figure}{\centering}

\newsubfloat{table}
\newsubfloat{figure}

%\setFloatBlockFor{chapter}

% ----- Listings -----
\lstset{basicstyle = \ttfamily, numberstyle = \tiny, stepnumber = 2}

% ----- Commands -----
% Unbind some things that are never useful.
    % Built-in probability operator:
\let\Pr\relax%
    % Slashed O:
\let\O\relax%

% Set up sentence-ending ellipses.
\def\ldotsplus{\mathinner{\ldotp\ldotp\ldotp\ldotp}}
\def\fourdots{\relax\ifmmode\ldotsplus\else$\m@th \ldotsplus\,$\fi}

% Use the nice phi.
\let\temp\phi
\let\phi\varphi
\let\varphi\temp


\DeclareMathOperator{\Pr}{\mathds{P}}
\DeclareMathOperator{\E}{\mathds{E}}
\DeclareMathOperator{\Var}{Var}
\DeclareMathOperator{\Cov}{Cov}
\DeclareMathOperator{\Cor}{Cor}
\DeclareMathOperator{\eul}{e}
\DeclareMathOperator{\1}{\mathbf{1}}
\DeclareMathOperator{\sign}{sign}
\DeclareMathOperator{\diag}{diag}
\DeclareMathOperator{\tr}{tr}
\DeclareMathOperator*{\argmin}{argmin}
\DeclareMathOperator*{\argmax}{argmax}
\DeclareMathOperator{\O}{\mathcal{O}}
%\DeclareMathOperator{\N}{N}
%\DeclareMathOperator{\F}{F}
%\DeclareMathOperator{\Wishart}{W}

\newcommand{\abs}[1]{\lvert#1\rvert}
\newcommand{\norm}[1]{\lVert#1\rVert}
\newcommand{\ceil}[1]{\lceil#1\rceil}
\newcommand{\floor}[1]{\lfloor#1\rfloor}
\newcommand{\df}{\,\mathrm{d}}
\newcommand{\pd}[2][]{\frac{\partial#1}{\partial#2}}
\newcommand{\dv}[2][]{\frac{\mathrm{d}#1}{\mathrm{d}#2}}
\newcommand{\inD}{\mathop{\overset{\mathrm{d}}{\longrightarrow}}}
\newcommand{\inP}{\mathop{\overset{\Pr}{\longrightarrow}}}
\newcommand{\qed}{\hfill \ensuremath{\square}}
\newcommand{\T}{^\mathsf{T}}
\newcommand{\C}{^\mathsf{C}}
\newcommand{\io}{\text{ i.o.}}
\newcommand{\ev}{\text{ ev.}}
\newcommand{\dist}[1]{\operatorname{#1}}
\newcommand{\iid}{\overset{\mathrm{iid}}{\sim}}
\newcommand{\on}[1]{\operatorname{\mathds{1}}\!\left\{#1\right\}}
\newcommand{\ind}{\protect\mathpalette{\protect\independenT}{\perp}}

\def\independenT#1#2{\mathrel{\rlap{$#1#2$}\mkern4mu{#1#2}}}


\pagestyle{name}
\title{STA 250: Homework 3}

\begin{document}
\chapter*{Report}
\begin{enumerate}
\item % ----- Q1 
The likelihood in the linkage problem is
    \[
    L(\lambda) = (2 + \lambda)^{125} (1 - \lambda)^{18+20} \lambda^{34},
    \]
so the log-likelihood is
    \[
    l(\lambda) = 125\ln(2 + \lambda) + 38\ln(1 - \lambda) + 34\ln(\lambda).
    \]
The derivative with respect to $\lambda$ is
    \[
    l'(\lambda) = \frac{125}{2 + \lambda} - \frac{38}{1 - \lambda}
    + \frac{34}{\lambda}
    \]
and the second derivative is
    \[
    l''(\lambda) 
    =
    -\frac{125}{(2 + \lambda)^2} - \frac{38}{(1 - \lambda)^2}
    -\frac{34}{\lambda^2}.
    \]
Since the likelihood has several discontinuities,
and the log-likelihood is only defined on $(0, 1)$,
the root-finding algorithms were only applied to $(-2, 0)$ and $(0, 1)$.
They converged to the values shown in table~\ref{tab:roots}.
The Newton-Raphson algorithm was started at $-1$ and $0.5$, respectively.
The tolerance was set to $10^{-10}$ in all cases.
We conclude that the MLE is $\hat{\lambda} \approx 0.6268$.
    \begin{table}[h]
    \begin{tabular}{lrrr}
    \toprule
    Method & Estimate & Iterations & Log-likelihood \\
    \midrule
    Bisection & -0.5507 & 41 & -- \\
    Newton-Raphson & -0.5507 & 5 & -- \\
    \midrule
    Bisection & 0.6268 & 39 & 67.38 \\
    Newton-Raphson & 0.6268 & 5 & 67.38 \\
    \bottomrule
    \end{tabular}
    \caption{
    Results of applying the root-finding algorithms on $(-2, 0)$ and $(0, 1)$,
    at top and bottom, respectively.
    }
    \label{tab:roots}
    \end{table}

\item % ----- Q2 
We are given the multivariate-t regression model
    \[
    Y_i \sim \dist{t}_p \bigl( \mu, \tfrac{1}{\tau_i}, \Psi \bigr),
    \qquad
    i = 1, \ldots, n,
    \]
which can also be written as
    \[
    Y_i \vert \tau_i, \theta 
    \sim
    \dist{N}_p \bigl( \mu, \tfrac{1}{\tau_i}\Psi \bigr)
    \qquad\text{and}\qquad
    \tau_i
    \sim
    \tfrac{1}{\nu} \dist{\chi^2}(\nu),
    \]
where $\theta = (\mu, \Psi)$ is the unknown parameter to be estimated,
and $\nu$ is a known constant.
Denote the observed data by $Y = (Y_1, \ldots, Y_n)$
and the missing data by $\tau = (\tau_1, \ldots, \tau_n)$
In each iteration, the EM algorithm maximizes
    \[
    Q(\theta \vert \theta^{(t)})
    =
    \E \bigl[ \ln p(\tau, Y \vert \theta) \big\vert Y, \theta^{(t)} \bigr],
    \]
so we must first compute the joint density.
We know that the density of $Y \vert \tau, \theta$ is
    \[
    p(Y \vert \tau, \theta)
    \propto
    \det(\Psi)^{-n/2} 
    \exp \biggl[
    -\frac{1}{2}\sum_{i=1}^n (Y_i - \mu)\T \tau_i \Psi^{-1} (Y_i - \mu) 
    \biggr].
    \]
Since $\tau$ is ancillary for $\theta$, its density has no effect on the
maximization of $Q$, and may be neglected.
Therefore
    \begin{align*}
    Q(\theta \vert \theta^{(t)})
    &=
    \E \biggl[
    n\ln [ \det(\Psi^{-1}) ]
    - \sum_{i=1}^n (Y_i - \mu)\T \tau_i \Psi^{-1} (Y_i - \mu)
    \Big\vert Y, \theta^{(t)}
    \biggr]
    \\ &=
    n\ln [ \det(\Psi^{-1}) ]
    - \sum_{i=1}^n (Y_i - \mu)\T \Psi^{-1} (Y_i - \mu)
    \E \bigl[ \tau_i \vert Y, \theta^{(t)} \bigr]
    \end{align*}
Then since
    \[
    \pd{\mu} Q(\theta \vert \theta^{(t)})
    =
    \sum_{i=1}^n 2\Psi^{-1} (Y_i - \mu)
    \E \bigl[ \tau_i \vert Y, \theta^{(t)} \bigr]
    \]
the maximizer in $\mu$ is
    \begin{equation}
    \label{eqn:mu}
    \mu^{(t+1)}
    =
    \frac {\sum_{i=1}^n Y_i \E \bigl[ \tau_i \vert Y, \theta^{(t)} \bigr]} 
    {\sum_{j=1}^n \E \bigl[ \tau_j \vert Y, \theta^{(t)} \bigr]}.
    \end{equation}
Now take the trace of $Q$, which is equal since $Q$ is scalar-valued,
to find
    \begin{align*}
    Q(\theta \vert \theta^{(t)})
    &=
    n\ln [ \det(\Psi^{-1}) ]
    - \sum_{i=1}^n \tr \bigl[
    (Y_i - \mu)\T \Psi^{-1} (Y_i - \mu)
    \bigr]
    \E \bigl[ \tau_i \vert Y, \theta^{(t)} \bigr]
    \\ &=
    n\ln [ \det(\Psi^{-1}) ]
    - \sum_{i=1}^n \tr \bigl[
    \Psi^{-1} (Y_i - \mu) (Y_i - \mu)\T
    \bigr]
    \E \bigl[ \tau_i \vert Y, \theta^{(t)} \bigr]
    \\ &=
    n\ln [ \det(\Psi^{-1}) ]
    - \tr \biggl[ \Psi^{-1} \sum_{i=1}^n
    (Y_i - \mu) (Y_i - \mu)\T
    \E \bigl[ \tau_i \vert Y, \theta^{(t)} \bigr]
    \biggr].
    \end{align*}
Based on the helpful facts provided with the assignment, 
the maximizer in $\Psi$ for a function of this form is the matrix
    \begin{equation}
    \label{eqn:psi}
    \Psi^{(t+1)}
    =
    \frac{1}{n} \sum_{i=1}^n
    (Y_i - \mu) (Y_i - \mu)\T
    \E \bigl[ \tau_i \vert Y, \theta^{(t)} \bigr].
    \end{equation}
Therefore $\theta^{(t+1)}$ can be computed by using the fact that
    \[
    \E \bigl[ \tau_i \vert Y, \theta^{(t)} \bigr]
    =
    \frac{
        \nu + p
    }{
        \nu + (Y_i - \mu^{(t)})\T [\Psi^{(t)}]^{-1} (Y_i - \mu^{(t)})
    }
    \]
to compute $\mu_{(t+1)}$ from equation~\ref{eqn:mu},
and then substituting $\mu = \mu_{(t+1)}$ into equation~\ref{eqn:psi}
in order to compute $\Psi^{(t+1)}$.

\item % ----- Q3 
We are given the hierarchical Poisson model
    \[
    Y_i \vert \lambda_i \sim \dist{Pois}(\lambda_i),
    \qquad
    \lambda_i \vert \beta \sim \dist{Gamma}(1, \beta),
    \qquad
    i = 1, \ldots, n.
    \]
Let $Y = (Y_1, \ldots, Y_n)$ denote the observed data.
    \begin{enumerate}
    \item \label{q:sa}
    For this part, let $\lambda = (\lambda_1, \ldots, \lambda_n)$ denote the
    missing data.
    In each iteration, the EM algorithm for this model maximizes
        \[
        Q_\mathrm{S}(\beta \vert \beta^{(t)})
        =
        \E \bigl[ 
        \ln p(\lambda, Y \vert \beta) \big\vert Y, \beta^{(t)} 
        \bigr].
        \]
    The parameter $\beta$ doesn't appear in the density of $Y \vert \lambda$,
    so we only need consider the density of $\lambda \vert \beta$,
    which is
        \[
        p(\lambda \vert \beta)
        =
        \beta^n \exp\biggl[ -\beta\sum_{i=1}^n \lambda_i \biggr].
        \]
    Thus the function $Q_\mathrm{S}$ is
        \begin{align*}
        Q_\mathrm{S}(\beta \vert \beta^{(t)})
        &=
        \E\biggl[
        n\ln\beta - \beta\sum_{i=1}^n 
        \lambda_i \Big\vert Y, \beta^{(t)}
        \biggr]
        \\ &=
        n\ln\beta - \beta\sum_{i=1}^n 
        \E\bigl[ \lambda_i \vert Y, \beta^{(t)} \bigr],
        \end{align*}
    and the derivative in $\beta$ is
        \[
        \pd{\beta} Q_\mathrm{S}(\beta \vert \beta^{(t)})
        =
        n\beta^{-1} - \sum_{i=1}^n
        \E\bigl[ \lambda_i \vert Y, \beta^{(t)} \bigr].
        \]
    Equating this to 0 reveals that the maximizer in $\beta$ is
        \[
        \beta_\mathrm{S}^{(t+1)}
        =
        \frac{n}{\sum_{i=1}^n \E\bigl[ \lambda_i \vert Y, \beta^{(t)} \bigr]}.
        \]
    In order to make this explicit, we must compute the expectation.
    The density of $\lambda_i \vert Y_i, \beta^{(t)}$ is
        \begin{align*}
        p(\lambda_i \vert Y_i, \beta^{(t)})
        &\propto
        p(Y_i \vert \lambda_i) p(\lambda_i \vert \beta^{(t)})
        \\ &\propto
        \eul^{-\lambda_i} \lambda_i^{Y_i} \eul^{-\beta^{(t)} \lambda_i}
        \\ &\propto
        \lambda^{Y_i} \eul^{-(\beta^{(t)} + 1)\lambda_i}
        \\ &\sim
        \dist{Gamma}(Y_i + 1, \beta^{(t)} + 1),
        \end{align*}
    and the expectation of a gamma distributions is given as
        \[
        \E\bigl[ \lambda_i \vert Y, \beta^{(t)} \bigr]
        =
        \frac{Y_i + 1}{\beta^{(t)} + 1}.
        \]
    Therefore the EM algorithm update is
        \begin{equation}
        \label{eqn:sa}
        \beta_\mathrm{S}^{(t+1)}
        =
        \frac{\beta^{(t)} + 1}{\bar{Y} + 1}.
        \end{equation}

    \item \label{q:aa}
    The cumulative distribution function for $\lambda_i \vert \beta$ is
        \[
        F(u \vert \beta)
        =
        1 - \eul^{-\beta u},
        \qquad
        u > 0,
        \]
    and its inverse is
        \[
        F^{-1}(v \vert \beta)
        =
        -\beta^{-1} \ln(1 - v),
        \qquad
        0 < v < 1.
        \]
    Define the random variables
        \[
        \eta_i
        =
        1 - \eul^{-\beta \lambda_i}
        =
        F(\lambda_i \vert \beta)
        \sim
        \dist{U}(0, 1),
        \qquad
        i = 1, \ldots, n,
        \]
    and let $\eta = (\eta_1, \ldots, \eta_n)$ denote the missing data for this
    part.
    Then an ancillary augmentation counterpart to the given sufficient
    augmentation is the model
        \[
        Y_i \vert \eta_i, \beta \sim 
        \dist{Poi} \bigl( -\beta^{-1} \ln(1 - \eta_i) \bigr),
        \qquad
        \eta_i \sim \dist{U}(0, 1).
        \]
    In each iteration, the EM algorithm for this model maximizes
        \[
        Q_\mathrm{A}(\beta \vert \beta^{(t)})
        =
        \E \bigl[ 
        \ln p(\eta, Y \vert \beta) \big\vert Y, \beta^{(t)} 
        \bigr].
        \]
    The parameter $\beta$ doesn't appear in the density for $\eta$,
    so we need only consider the density of $Y \vert \eta, \beta$,
    which is
        \[
        p(Y \vert \eta, \beta)
        \propto
        \exp \biggl[ \beta^{-1}\sum_{i=1}^n \ln(1 - \eta_i) \biggr]
        \beta^{-n\bar{Y}},
        \]
    where all terms that don't depend on $\beta$ have been omitted.
    Then the function $Q_\mathrm{A}$ is
        \begin{align*}
        Q_\mathrm{A}(\beta \vert \beta^{(t)})
        &=
        \E \biggl[ 
        \beta^{-1}\sum_{i=1}^n \ln(1 - \eta_i) - n\bar{Y}\ln\beta
        \Big\vert Y, \beta^{(t)} \biggr]
        \\ &=
        \beta^{-1}\sum_{i=1}^n 
        \E\bigl[ \ln(1 - \eta_i) \vert Y, \beta^{(t)} \bigr]
        - n\bar{Y}\ln\beta
        \\ &=
        -\beta^{(t)}\beta^{-1}\sum_{i=1}^n 
        \E\bigl[ -(\beta^{(t)})^{-1}\ln(1 - \eta_i) \vert Y, \beta^{(t)} \bigr]
        - n\bar{Y}\ln\beta
        \\ &=
        -\beta^{(t)}\beta^{-1}
        \sum_{i=1}^n \E\bigl[ \lambda_i \vert Y, \beta^{(t)} \bigr]
        - n\bar{Y}\ln\beta,
        \end{align*}
    and the derivative in $\beta$ is
        \[
        \pd{\beta} Q_\mathrm{A}(\beta \vert \beta^{(t)})
        =
        \beta^{(t)}\beta^{-2}
        \sum_{i=1}^n \E\bigl[ \lambda_i \vert Y, \beta^{(t)} \bigr]
        - n\bar{Y}\beta^{-1}.
        \]
    Equating this to 0, we have
        \[
        0 
        = 
        \beta^{(t)} \sum_{i=1}^n \E\bigl[ \lambda_i \vert Y, \beta^{(t)} \bigr]
        - n\bar{Y}\beta,
        \]
    so the maximizer in $\beta$ is
        \begin{align} \label{eqn:aa}
        \notag
        \beta_\mathrm{A}^{(t+1)}
        &=
        \frac{\beta^{(t)}}{\bar{Y}} \cdot \frac{1}{n}
        \sum_{i=1}^n \E\bigl[ \lambda_i \vert Y, \beta^{(t)} \bigr]
        \\ &=\notag
        \frac{\beta^{(t)}}{\bar{Y}} \cdot 
        \frac{\bar{Y} + 1}{\beta^{(t)} + 1}
        \\ &=
        \frac{1 + \bar{Y}^{-1}}{1 + 1 / \beta^{(t)}}.
        \end{align}
    This is also the EM algorithm update.

    \item
    We will use the ancillary augmentation as the first scheme and the
    sufficient augmentation as the second scheme in the interwoven EM,
    with the notations from both part~\ref{q:sa} and part~\ref{q:aa}.
    Define
        \[
        \beta_\mathrm{I}^{(t+0.5)}
        =
        \beta_\mathrm{S}^{(t+1)}
        =
        \frac{\beta^{(t)} + 1}{\bar{Y} + 1}
        \]
    from the SA update (equation~\ref{eqn:sa}).
    In each iteration, the IEM algorithm maximizes
        \begin{align*}
        Q_\mathrm{I}(\beta \vert \beta^{(t)})
        &=
        \E_\mathrm{S} \biggl[
        \E_\mathrm{A} \Bigl[
        \ln p(\eta, Y \vert \beta) 
        \big\vert Y, \lambda, \beta_\mathrm{I}^{(t+0.5)} \Bigl] 
        \Big\vert Y, \beta^{(t)} \biggr]
        \\ &=
        \E_\mathrm{S} \biggl[
        \E_\mathrm{A} \Bigl[
        \beta^{-1}\sum_{i=1}^n \ln(1 - \eta_i) - n\bar{Y}\ln\beta
        \big\vert Y, \lambda, \beta_\mathrm{I}^{(t+0.5)} \Bigl] 
        \Big\vert Y, \beta^{(t)} \biggr]
        \\ &=
        \E_\mathrm{S} \biggl[
            \beta^{-1}\sum_{i=1}^n 
            \E_\mathrm{A} \bigl[
                \ln(1 - \eta_i) 
            \vert Y, \lambda, \beta_\mathrm{I}^{(t+0.5)} \bigl]
            - n\bar{Y}\ln\beta
        \Big\vert Y, \beta^{(t)} \biggr]
        \\ &=
        \E_\mathrm{S} \biggl[
            -\beta_\mathrm{I}^{(t+0.5)} \beta^{-1}\sum_{i=1}^n 
            \E_\mathrm{A} \bigl[
                \lambda_i
            \vert Y, \lambda, \beta_\mathrm{I}^{(t+0.5)} \bigl]
            - n\bar{Y}\ln\beta
        \Big\vert Y, \beta^{(t)} \biggr]
        \\ &=
        \E_\mathrm{S} \biggl[
            -\beta_\mathrm{I}^{(t+0.5)} \beta^{-1}\sum_{i=1}^n 
            \lambda_i
            - n\bar{Y}\ln\beta
        \Big\vert Y, \beta^{(t)} \biggr]
        \\ &=
        -\beta_\mathrm{I}^{(t+0.5)} \beta^{-1}\sum_{i=1}^n 
        \E_\mathrm{S} [
            \lambda_i
        \vert Y, \beta^{(t)} ]
        - n\bar{Y}\ln\beta.
        \end{align*}
    The derivative in $\beta$ is
        \[
        \pd{\beta} Q_\mathrm{I}(\beta \vert \beta^{(t)})
        =
        \beta_\mathrm{I}^{(t+0.5)} \beta^{-2}\sum_{i=1}^n 
        \E_\mathrm{S} [
            \lambda_i
        \vert Y, \beta^{(t)} ]
        - n\bar{Y}\beta^{-1}.
        \]
    Equating this to 0, we have
        \[
        0
        =
        \beta_\mathrm{I}^{(t+0.5)} 
        \sum_{i=1}^n \E_\mathrm{S} [
            \lambda_i
        \vert Y, \beta^{(t)} ]
        - n\bar{Y}\beta,
        \]
    so the maximizer in $\beta$ is
        \begin{align*}
        \beta_\mathrm{I}^{(t+1)}
        &=
        \beta_\mathrm{I}^{(t+0.5)} \cdot
        \frac{
            \sum_{i=1}^n \E_\mathrm{S} [
                \lambda_i
            \vert Y, \beta^{(t)} ]
        }{
            n\bar{Y}
        }
        \\ &=
        \frac{\beta^{(t)} + 1}{\bar{Y} + 1} \cdot
        \frac{\bar{Y} + 1}{(\beta^{(t)} + 1)\bar{Y}}
        \\&=
        \bar{Y}^{-1}
        \end{align*}
    Thus the IEM converges in one iteration.

    \item
    The single-observation likelihood for the given model is
        \begin{align*}
        p(Y_i \vert \beta)
        &=
        \int_0^\infty\!\! 
        p(\lambda_i, Y_i \vert \beta) 
        \df{\lambda_i}
        \\ &=
        \int_0^\infty\!\! 
        p(Y_i \vert \lambda_i) p(\lambda_i \vert \beta)
        \df{\lambda_i}
        \\ &=
        \frac{\beta}{Y_i!}
        \int_0^\infty\!\!
        \eul^{-\lambda_i} \lambda_i^{Y_i} 
        \eul^{-\beta \lambda_i}
        \df{\lambda_i}
        \\ &=
        \frac{\beta}{Y_i!}
        \int_0^\infty\!\!
        \lambda^{Y_i} \eul^{-(\beta + 1)\lambda_i}
        \df{\lambda_i},
    \intertext{
    and by noticing that the integrand is part of a 
    $\dist{Gamma(Y_i + 1, \beta + 1)}$ density,
    }
        &=
        \frac{\beta}{Y_i!} \cdot
        Y_i! \Bigl( \frac{1}{\beta + 1} \Bigr)^{Y_i + 1}
        \\ &=
        \beta \Bigl( \frac{1}{\beta + 1} \Bigr)^{Y_i + 1}.
        \end{align*}
    It follows by independence that the observed data likelihood is
        \[
        p(Y \vert \beta)
        =
        \beta^n \Bigl( \frac{1}{\beta + 1} \Bigr)^{n(\bar{Y} + 1)}
        \]
    and hence the log likelihood is
        \[
        l(\beta)
        =
        \ln p(Y \vert \beta)
        =
        n\ln\beta - n(\bar{Y} + 1)\ln(\beta + 1).
        \]
    Then
        \[
        \pd{\beta} l(\beta)
        =
        n\beta^{-1} - n(\bar{Y} + 1)(\beta + 1)^{-1}.
        \]
    Equating this to 0 leads to
        \begin{align*}
        0
        &=
        \beta^{-1} - (\bar{Y} + 1)(\beta + 1)^{-1}
        \\ &=
        (\beta + 1) - (\bar{Y} + 1)\beta
        \\ &=
        1 - \bar{Y}\beta
        \\
        \beta
        &=
        \bar{Y}^{-1}.
        \end{align*}
    Therefore the MLE for $\beta$ is $\hat{\beta} = \bar{Y}^{-1}$.

    \item
    For the sufficient augmentation scheme, the update operator is
        \[
        M_\mathrm{S}(\beta)
        =
        \frac{1 + \beta}{1 + \bar{Y}},
        \]
    based on equation~\ref{eqn:sa}. The Taylor expansion about $\bar{Y}^{-1}$
    is
        \begin{align*}
        M_\mathrm{S}(\beta)
        &\approx
        \frac{1 + \bar{Y}^{-1}}{1 + \bar{Y}}
        + \frac{1}{1 + \bar{Y}} (\beta - \bar{Y}^{-1})
        \\ &\approx
        \bar{Y} + \frac{1}{1 + \bar{Y}} (\beta - \bar{Y}^{-1}).
        \end{align*}
    It follows that
        \[
        \beta_\mathrm{S}^{(t+1)} - \bar{Y}^{-1}
        \approx
        \frac{1}{1 + \bar{Y}} (\beta^{(t)} - \bar{Y}^{-1})
        \]
    so the rate of convergence is approximately
        \[
        \rho_\mathrm{S}
        =
        \frac{1}{1 + \bar{Y}}.
        \]
    For the ancillary augmentation scheme, the update operator is
        \begin{align*}
        M_\mathrm{A}(\beta)
        &=
        \frac{1 + \bar{Y}^{-1}}{1 + \beta^{-1}},
        \end{align*}
    based on equation~\ref{eqn:aa}.
    The Taylor expansion about $\bar{Y}^{-1}$ is
        \begin{align*}
        M_\mathrm{A}(\beta) - \bar{Y}^{-1}
        &\approx
        \frac{1 + \bar{Y}^{-1}}{1 + \bar{Y}}
        + \biggl[
        \frac{(1 + \bar{Y}^{-1})\beta^{-2}}{(1 + \beta^{-1})^2}
        \biggr]_{\beta = \bar{Y}^{-1}}
        (\beta - \bar{Y}^{-1})
        \\ &\approx
        \bar{Y}
        + \biggl[
        \frac{1 + \bar{Y}^{-1}} {(1 + \beta)^2}
        \biggr]_{\beta = \bar{Y}^{-1}}
        (\beta - \bar{Y}^{-1})
        \\ &\approx
        \bar{Y}
        + \frac{1} {1 + \bar{Y}^{-1}}
        (\beta - \bar{Y}^{-1}),
        \end{align*}
    so it follows that
        \[
        \beta_\mathrm{A}^{(t+1)} - \bar{Y}^{-1}
        \approx
        \frac{1} {1 + \bar{Y}^{-1}}
        (\beta^{(t)} - \bar{Y}^{-1}).
        \]
    Thus the rate of convergence is approximately
        \[
        \rho_\mathrm{A}
        =
        \frac{1} {1 + \bar{Y}^{-1}}.
        \]
    Smaller convergence rates are better, so we can solve the inequality
        \begin{align*}
        \rho_\mathrm{S}
        =
        \frac{1}{1 + \bar{Y}}
        &>
        \frac{1}{1 + \bar{Y}^{-1}}
        =
        \rho_\mathrm{A}
        \\
        1 + \bar{Y} &< 1 + \bar{Y}^{-1}
        \\
        \bar{Y} + \bar{Y}^2 &< \bar{Y} + 1
        \\
        \bar{Y}^2 &< 1
        \\
        \bar{Y} &< 1
        \end{align*}
    to see that the ancillary augmentation scheme is superior when 
    $\bar{Y} < 1$, 
    while the sufficient augmentation scheme is superior when $\bar{Y} > 1$.
    That being said, the interwoven scheme converges in one iteration,
    so it's superior to both of the other schemes,
    regardless of the value of $\bar{Y}$.

    \end{enumerate}

\end{enumerate}

\clearpage
\chapter*{Source Code}
\hrule
\lstinputlisting{../root_finders.py}

\end{document}
