
\documentclass[article]{memoir}

% ----- Packages -----
\usepackage{amsmath}
\usepackage{amsfonts}
\usepackage{amssymb}
% Must come after amsmath.
%\usepackage{amsthm}

\usepackage{mathtools}
\usepackage{subdepth}
\usepackage{dsfont}
\usepackage{flafter}
\usepackage{graphicx}
\usepackage{listings}

% ----- Page -----
\setlrmarginsandblock{1in}{1in}{*}
\setulmarginsandblock{1in}{1in}{*}
\checkandfixthelayout

\makepagestyle{name}
    \makeevenhead{name}{\textsl\thetitle}{}
    {\textsl{\thepage\ of \thelastpage\ --- Nick Ulle}}
    \makeoddhead{name}{\textsl\thetitle}{}
    {\textsl{\thepage\ of \thelastpage\ --- Nick Ulle}}
\makepagestyle{no-name}
    \makeevenhead{no-name}{\textsl\thetitle}{}
    {\textsl{\thepage\ of \thelastpage}}
    \makeoddhead{no-name}{\textsl\thetitle}{}
    {\textsl{\thepage\ of \thelastpage}}

% ----- Floats -----
\changecaptionwidth
\captionwidth{0.6\textwidth}
\setfloatadjustment{table}{\centering}
\setfloatadjustment{figure}{\centering}

\newsubfloat{table}
\newsubfloat{figure}

%\setFloatBlockFor{chapter}

% ----- Listings -----
\lstset{basicstyle = \ttfamily, numberstyle = \tiny, stepnumber = 2}

% ----- Commands -----
% Unbind some things that are never useful.
    % Built-in probability operator:
\let\Pr\relax%
    % Slashed O:
\let\O\relax%

% Set up sentence-ending ellipses.
\def\ldotsplus{\mathinner{\ldotp\ldotp\ldotp\ldotp}}
\def\fourdots{\relax\ifmmode\ldotsplus\else$\m@th \ldotsplus\,$\fi}

% Use the nice phi.
\let\temp\phi
\let\phi\varphi
\let\varphi\temp


\DeclareMathOperator{\Pr}{\mathds{P}}
\DeclareMathOperator{\E}{\mathds{E}}
\DeclareMathOperator{\Var}{Var}
\DeclareMathOperator{\Cov}{Cov}
\DeclareMathOperator{\Cor}{Cor}
\DeclareMathOperator{\eul}{e}
\DeclareMathOperator{\1}{\mathbf{1}}
\DeclareMathOperator{\sign}{sign}
\DeclareMathOperator{\diag}{diag}
\DeclareMathOperator{\tr}{tr}
\DeclareMathOperator*{\argmin}{argmin}
\DeclareMathOperator*{\argmax}{argmax}
\DeclareMathOperator{\O}{\mathcal{O}}
%\DeclareMathOperator{\N}{N}
%\DeclareMathOperator{\F}{F}
%\DeclareMathOperator{\Wishart}{W}

\newcommand{\abs}[1]{\lvert#1\rvert}
\newcommand{\norm}[1]{\lVert#1\rVert}
\newcommand{\ceil}[1]{\lceil#1\rceil}
\newcommand{\floor}[1]{\lfloor#1\rfloor}
\newcommand{\df}{\,\mathrm{d}}
\newcommand{\pd}[2][]{\frac{\partial#1}{\partial#2}}
\newcommand{\dv}[2][]{\frac{\mathrm{d}#1}{\mathrm{d}#2}}
\newcommand{\inD}{\mathop{\overset{\mathrm{d}}{\longrightarrow}}}
\newcommand{\inP}{\mathop{\overset{\Pr}{\longrightarrow}}}
\newcommand{\qed}{\hfill \ensuremath{\square}}
\newcommand{\T}{^\mathsf{T}}
\newcommand{\C}{^\mathsf{C}}
\newcommand{\io}{\text{ i.o.}}
\newcommand{\ev}{\text{ ev.}}
\newcommand{\dist}[1]{\operatorname{#1}}
\newcommand{\iid}{\overset{\mathrm{iid}}{\sim}}
\newcommand{\on}[1]{\operatorname{\mathds{1}}\!\left\{#1\right\}}
\newcommand{\ind}{\protect\mathpalette{\protect\independenT}{\perp}}

\def\independenT#1#2{\mathrel{\rlap{$#1#2$}\mkern4mu{#1#2}}}


\pagestyle{name}
\title{STA 250: Homework 4}

\begin{document}
\chapter*{Part 1}
The function \texttt{gpu\_trunc\_norm()} allows sampling from the truncated
normal distribution on the GPU.
When the truncation region is a left or right tail, it uses the methods
described in Robert~(2009).
In all other cases, it uses rejection sampling from a normal distribution.
Maximum iterations are capped at 100 to guarantee a finite runtime.
It is assumed that random number generators will be pre-initialized on the
graphics device before, and convenience functions are included to perform this
initialization and an eventual de-initialization.
The sampling function could be improved by adopting the other procedures
described in Robert~(2009),
by allowing user control of the maximum number of iterations,
and by outputting a warning when the maximum number of iterations is achieved.
As written, the function produces results that are not reproducible.
A strategy where a relatively small number of threads are run,
each sampling multiple values using their own private random number generator,
would make the results reproducible.
This would also improve scalability, since the maximum number of samples drawn
in one call would no longer be restricted by hardware limitations.

Sampling from the truncated normal distribution on the CPU is provided by
the correspondingly named function \texttt{cpu\_trunc\_norm()}.
The details of the algorithm are identical,
with random number generation provided by the Mersenne Twister implementation
included in the recent C++11 standard.
A convenience function is included for setting the seed,
and results from this sampling function are reproducible.

A Python script is included to perform verifications that these functions
produce samples with the correct means based on theory.
They performed well when the truncation region included the mean or was
one-sided, but as expected, did poorly when the truncation region was
two-sided and did not include the mean.
Again, this could be remedied by adopting the full algorithm described in
Robert~(2009).
Timings of these functions when sampling $n = 10^k$ values from
$\dist{N}(2, 1)$ truncated to $(0, 1.5)$, for each of $k = 1, \ldots, 8$,
are shown in figure~.
The CPU function performs better for $k = 1, 2$ but the GPU function performs
better otherwise.
This is a little surprising, because the CPU function is written in C++.
PyCUDA appears to do a superb job at minimizing the overhead of using the GPU.

    \begin{figure}[p]
    \includegraphics[width = 0.9\textwidth]{res/timings.png}
    \caption{
        Comparison of CPU and GPU timings for sampling $10^k$ values.
    }
    \label{fig:timings}
    \end{figure}

\chapter*{Part 2}
The probit model is
    \begin{gather*}
    Y_i \vert Z_i = \on{Z_i > 0}
    \\
    Z_i \vert \beta \sim \dist{N}(x_i\T\beta, 1)
    \\
    \beta \sim \dist{N}_p(\beta_0, \Sigma_0)
    \end{gather*}
for $\beta_0$ a $p \times 1$ vector.
Define $X = (x_1\T, \ldots, x_n\T)$, so that that
    \[
    Z \vert \beta \sim \dist{N}_n(X\beta, \eye_n).
    \]
It follows that the posterior density of $\beta$ is
    \begin{align*}
        p(\beta \vert Z, Y)
    &\propto
        p(Y \vert Z, \beta) p(Z \vert \beta) p(\beta)
    \\ &\propto
        \exp \biggl[ -\frac{1}{2} (Z - X\beta)\T (Z - X\beta) \biggr]
        \exp \biggl[ -\frac{1}{2} (\beta - \beta_0)\T \Sigma_0^{-1}
        (\beta - \beta_0) \biggr].
    \end{align*}
The exponential is quadratic in $\beta$, so this is a multivariate normal
density, and since
    \begin{align*}
    &
        Z\T Z - 2\beta\T X\T Z + \beta\T X\T X \beta
        + \beta\T \Sigma_0^{-1} \beta - 2 \beta\T \Sigma_0^{-1} \beta_0
        + \beta_0\T \Sigma_0^{-1} \beta_0
    \\ &=
        \beta\T (X\T X + \Sigma_0^{-1}) \beta
        - 2\beta\T (X\T Z + \Sigma_0^{-1})
        + \ldots
    \end{align*}
the mean and variance are
    \begin{gather*}
    \beta_0^* = (X\T X + \Sigma_0^{-1})^{-1} (X\T Z + \Sigma_0^{-1})
    \\
    \Sigma_0^* = (X\T X + \Sigma_0^{-1})^{-1}.
    \end{gather*}
Combined with the fact that
    \begin{gather*}
    Z_i \vert Y_i = 0, \beta \sim \dist{N}(x_i\T \beta, 1) \1_{(0, \infty)}
    \\
    Z_i \vert Y_i = 1, \beta \sim \dist{N}(x_i\T \beta, 1) \1_{(-\infty, 0)},
    \end{gather*}
this suggests we can sample from $\beta \vert Y$ by using a Gibbs sampler to
sample from $Z, \beta \vert Y$.

The probit MCMC was implemented as the function \texttt{probit\_mcmc()}.
This function includes a parameter controlling whether the GPU or CPU should
be used when sampling from the truncated normal distribution.
It was used to calculate posterior means for datasets 1 through 5,
and the results were close to the true means in all cases,
with sharper results for the larger datasets.
Timings of the GPU and CPU code are presented in table~\ref{tab:timings_mcmc}
and figure~\ref{fig:timings_mcmc}.
Both methods exhibit runtimes that are linear in the sample size
(provided the sample size is large enough to amortize fixed overhead),
but the CPU code is only competetive with the GPU code for the first
and smallest dataset.
On the large datasets, the GPU code is about 5 times faster
than the CPU code, which is a substantial speedup.

    \begin{table}[h]
    \begin{tabular}{lrr}
    \toprule
    Dataset & GPU & CPU \\
    \midrule
    \input{res/timings_mcmc.tex}
    \bottomrule
    \end{tabular}
    \caption{
        GPU and CPU timings, in seconds, for probit MCMC.
    }
    \label{tab:timings_mcmc}
    \end{table}

This was an especially interesting assignment,
because I not only learned to program in CUDA C,
but also learned about the changes in the new C++11 standard
and how to call C/C++ code from Python.

    \begin{figure}[p]
    \includegraphics[width = 0.9\textwidth]{res/timings_mcmc.png}
    \caption{
        Comparison of CPU and GPU timings for probit MCMC.
    }
    \label{fig:timings_mcmc}
    \end{figure}

\clearpage
\chapter*{Source Code}
\hrule
\lstinputlisting{../src/gpu_trunc_norm.py}
\hrule
\lstinputlisting{../src/gpu_trunc_norm.cu}
\hrule
\lstinputlisting{../src/cpu_trunc_norm.py}
\hrule
\lstinputlisting{../src/cpu_trunc_norm.cpp}
\hrule
\lstinputlisting{../src/trunc_norm_diag.py}
\hrule
\lstinputlisting{../src/probit_mcmc.py}
\hrule
\lstinputlisting{../src/postproc_timings.R}

\end{document}
