
\documentclass[article]{memoir}

% ----- Packages -----
\usepackage{amsmath}
\usepackage{amsfonts}
\usepackage{amssymb}
% Must come after amsmath.
%\usepackage{amsthm}

\usepackage{mathtools}
\usepackage{subdepth}
\usepackage{dsfont}
\usepackage{flafter}
\usepackage{graphicx}
\usepackage{listings}

% ----- Page -----
\setlrmarginsandblock{1in}{1in}{*}
\setulmarginsandblock{1in}{1in}{*}
\checkandfixthelayout

\makepagestyle{name}
    \makeevenhead{name}{\textsl\thetitle}{}
    {\textsl{\thepage\ of \thelastpage\ --- Nick Ulle}}
    \makeoddhead{name}{\textsl\thetitle}{}
    {\textsl{\thepage\ of \thelastpage\ --- Nick Ulle}}
\makepagestyle{no-name}
    \makeevenhead{no-name}{\textsl\thetitle}{}
    {\textsl{\thepage\ of \thelastpage}}
    \makeoddhead{no-name}{\textsl\thetitle}{}
    {\textsl{\thepage\ of \thelastpage}}

% ----- Floats -----
\changecaptionwidth
\captionwidth{0.6\textwidth}
\setfloatadjustment{table}{\centering}
\setfloatadjustment{figure}{\centering}

\newsubfloat{table}
\newsubfloat{figure}

%\setFloatBlockFor{chapter}

% ----- Listings -----
\lstset{basicstyle = \ttfamily, numberstyle = \tiny, stepnumber = 2}

% ----- Commands -----
% Unbind some things that are never useful.
    % Built-in probability operator:
\let\Pr\relax%
    % Slashed O:
\let\O\relax%

% Set up sentence-ending ellipses.
\def\ldotsplus{\mathinner{\ldotp\ldotp\ldotp\ldotp}}
\def\fourdots{\relax\ifmmode\ldotsplus\else$\m@th \ldotsplus\,$\fi}

% Use the nice phi.
\let\temp\phi
\let\phi\varphi
\let\varphi\temp


\DeclareMathOperator{\Pr}{\mathds{P}}
\DeclareMathOperator{\E}{\mathds{E}}
\DeclareMathOperator{\Var}{Var}
\DeclareMathOperator{\Cov}{Cov}
\DeclareMathOperator{\Cor}{Cor}
\DeclareMathOperator{\eul}{e}
\DeclareMathOperator{\1}{\mathbf{1}}
\DeclareMathOperator{\sign}{sign}
\DeclareMathOperator{\diag}{diag}
\DeclareMathOperator{\tr}{tr}
\DeclareMathOperator*{\argmin}{argmin}
\DeclareMathOperator*{\argmax}{argmax}
\DeclareMathOperator{\O}{\mathcal{O}}
%\DeclareMathOperator{\N}{N}
%\DeclareMathOperator{\F}{F}
%\DeclareMathOperator{\Wishart}{W}

\newcommand{\abs}[1]{\lvert#1\rvert}
\newcommand{\norm}[1]{\lVert#1\rVert}
\newcommand{\ceil}[1]{\lceil#1\rceil}
\newcommand{\floor}[1]{\lfloor#1\rfloor}
\newcommand{\df}{\,\mathrm{d}}
\newcommand{\pd}[2][]{\frac{\partial#1}{\partial#2}}
\newcommand{\dv}[2][]{\frac{\mathrm{d}#1}{\mathrm{d}#2}}
\newcommand{\inD}{\mathop{\overset{\mathrm{d}}{\longrightarrow}}}
\newcommand{\inP}{\mathop{\overset{\Pr}{\longrightarrow}}}
\newcommand{\qed}{\hfill \ensuremath{\square}}
\newcommand{\T}{^\mathsf{T}}
\newcommand{\C}{^\mathsf{C}}
\newcommand{\io}{\text{ i.o.}}
\newcommand{\ev}{\text{ ev.}}
\newcommand{\dist}[1]{\operatorname{#1}}
\newcommand{\iid}{\overset{\mathrm{iid}}{\sim}}
\newcommand{\on}[1]{\operatorname{\mathds{1}}\!\left\{#1\right\}}
\newcommand{\ind}{\protect\mathpalette{\protect\independenT}{\perp}}

\def\independenT#1#2{\mathrel{\rlap{$#1#2$}\mkern4mu{#1#2}}}


\pagestyle{name}
\title{STA 250: Final Project}
\author{Nicholas Ulle}

\captionsetup[sub]{font = scriptsize}

\begin{document}
\maketitle
\clearpage
\chapter*{Introduction}
Markov chain Monte Carlo methods are especially useful in situations where
the distribution of interest has a density that is difficult or impossible to
compute analytically, because they avoid direct integration and
instead rely on central limit theorem consistency results.
However, for these results to be valid, it's essential that the method used
explores the state space effectively, leaving no probability mass unrecognized.
In this setting, it's also unlikely that the practitioner will have much 
advance knowledge about the distribution of interest, so checking how well
the state space was explored is difficult.
Particularly problematic are multi-modal distributions with relatively
isolated modes, because standard MCMC methods tend to become trapped at one
of the modes and never discover the other(s).

Parallel-tempering attempts to address this problem.
The key idea is that if the distribution of interest has density proportional
to $f$, then for $\tau > 0$, the distribution with density proportional to 
$f^{1/\tau}$, will be flatter, with less isolated modes.
Thus a chain converging to the distribution governed by $f^{1/\tau}$ will
move about the state space with relatively less difficulty.
This can be utilized by running chains converging to these two distributions
in parallel, and periodically proposing a swap of their states.
That is, the state of each chain is periodically used as a proposal for the 
other, in a Metropolis update step.
As a result, the chains become coupled, but it can be shown that they still
converge to the correct stationary distributions.
In order to improve the chances of a swap being accepted, while still gleaning
the benefits of a flat $f^{1/\tau}$, several values $\tau_1, \ldots, \tau_m$
may be used, with $m$ chains running in parallel. Swaps can then be performed
between adjacent chains, in hope that the effect of the flattest chain will
trickle down to the target chain.

Drawing on a physics-based interpretation, the values $\tau_1, \ldots, \tau_m$
are termed temperatures, and their range is known as a temperature ladder.
The corresponding densities, usually written as
    \[
    f^{1/\tau} = \eul^{U / \tau},
    \]
are called replicas, while the function $U$ is called the energy potential
function. 
The target distribution can be regarded as having temperature $\tau_0 = 1$,
while the flatter replicas have higher temperatures.

\chapter*{Implementation}
A univariate version of the parallel tempering algorithm was implemented in
Python.
At each iteration, every chain is first updated according to a Metropolis
algorithm update. 
This is always followed by a swap proposal between two of the chains,
which is counted as a successive iteration only when it involves the target
distribution's chain.
The implementation was designed to support any target distribution that can
be defined as a univariate Python function, and any symmetric proposal
distribution that satisfies the same.
The acceptance rate for the target distribution's chain is tracked,
although acceptance rates for the replica chains are not.
For the sake of efficiency, only the history of the target distribution's
chain is saved.

A standalone Metropolis algorithm was also implemented, for comparison.
Similar features are supported. Additional tools include several probability
densities, and a demonstration script, 
the results of which are described in the following section.
The Python module \texttt{Pymc} was considered for post-sampling diagnostics
and plots, but lack of documentation on how to use it with an externally
generated chain led to the R package \texttt{coda} being used instead.

\chapter*{Results}

First, the correctness of the implementation was checked by running both the 
Metropolis algorithm and the parallel tempering algorithm with a
$\dist{Laplace(0, 1)}$ distribution as the target.
This distribution is unimodal and symmetric, making it an extremely easy
target.
Each algorithm was run for $10^5$ iterations, with a burn-in period of
$5000$ iterations ($5$\%).
A $\dist{N}(x^{(t)}, 9)$ proposal distribution was used for Metropolis updates,
where $x^{(t)}$ denotes the current state of the chain.
The variance $9$ was selected because it produced favorable acceptance rates.
Four different temperature ladders were considered for the parallel
tempering algorithm.
The first two of these are linear, in the sense that they follow the pattern
    \[
    1, 1 + \lambda, 1 + 2\lambda, 1 + 3\lambda, \ldots, 1 + (m - 1)\lambda
    \]
for some constant $\lambda$.
The second two are geometric, meaning they follow the pattern
    \[
    1, \lambda, \lambda^2, \lambda^3, \ldots, \lambda^{m - 1}.
    \]
For all cases, $m = 5$ replicas were used.

    \begin{table}[ht]
    \begin{tabular}{lrrrrr}
    \toprule
    & & \multicolumn{4}{c}{Parallel Tempering} \\
    \cmidrule(l){3-6}
    & Metropolis & Linear $(1)$ & Linear $(2)$ 
    & Geometric $(1.5)$ & Geometric $(2)$ \\
    \midrule
    \input{res/demo_laplace/diagnostics.tex}
    \bottomrule
    \end{tabular}
    \caption{
    Diagnostics for the $\dist{Laplace}(0, 1)$ example.
    }
    \label{tab:laplace_diag}
    \end{table}

Diagnostics for the resulting samples are shown in 
table~\ref{tab:laplace_diag}.
Although this was a relatively easy target distribution,
the superiority of the parallel tempering algorithm over the standard
Metropolis algorithm is already evident.
The lag-1 autocorrelations for the parallel-tempering samples are smaller,
and the effective sample sizes are larger.
On top of this, the $p$-value of the Geweke diagnostic, which assumes
stationarity as its null hypothesis, is substantially larger for the
parallel-tempering samples.
Overall, the parallel tempering chains seem to be mixing better and approaching
stationary much faster than the Metropolis chain.
However, among the different choices of temperature ladder for the
parallel tempering, none stand out above the others.
Density plots for the samples are presented in figure~\ref{fig:laplace_dens}.
Despite the diagnostic differences between the chains, there is not much
difference visible in the plots; all of the samples approximate the target
distribution reasonably well.

    \begin{figure}[h]
    \begin{subfigure}[b]{0.3\textwidth}
        \includegraphics[width = \textwidth]{res/demo_laplace/density_00.png}
        \caption{Metropolis}
    \end{subfigure}
    \begin{subfigure}[b]{0.3\textwidth}
        \includegraphics[width = \textwidth]{res/demo_laplace/density_02.png}
        \caption{PT -- Linear $(2)$}
    \end{subfigure}
    \begin{subfigure}[b]{0.3\textwidth}
        \includegraphics[width = \textwidth]{res/demo_laplace/density_04.png}
        \caption{PT -- Geometric $(2)$}
    \end{subfigure}
    \caption{
    Density plots for the $\dist{Laplace}(0, 1)$ example.
    Plots for the other parallel tempering samples were similar.
    The true density is shown dotted.
    }
    \label{fig:laplace_dens}
    \end{figure}

A slightly more difficult target distribution was used for the next example,
although it may still be within reach of a carefully-tuned and long-running
Metropolis algorithm.
The distribution used was a mixture of a $\dist{N}(-6, 0.25)$ distribution
and a $\dist{N}(8, 1)$ distribution, with respective proportions $20$\% and
$80$\%.
Most other settings were kept the same as in the first example,
including the choices of temperature ladder.

    \begin{table}[ht]
    \begin{tabular}{lrrrrr}
    \toprule
    & & \multicolumn{4}{c}{Parallel Tempering} \\
    \cmidrule(l){3-6}
    & Metropolis & Linear $(1)$ & Linear $(2)$ 
    & Geometric $(1.5)$ & Geometric $(2)$ \\
    \midrule
    \input{res/demo_01/diagnostics.tex}
    \bottomrule
    \end{tabular}
    \caption{
    Diagnostics for the gentle normal mixture example.
    }
    \label{tab:01_diag}
    \end{table}

Table~\ref{tab:01_diag} shows the corresponding diagnostics.
Once again, the Metropolis algorithm performs poorly next to the parallel
tempering algorithm.
However, a more interesting result is also apparent.
The linear temperature ladder with $\lambda = 2$ does substantially better
on the Geweke diagnostic than any of the others, although the effective
sample size is actually lower than the others.
On the other hand, the linear temperature ladder with $\lambda = 1$ has the
highest effective sample size by a small margin.
The density plots in figure~\ref{fig:01_density} indicate that in this example,
the Metropolis algorithm also did a poor job of exploring the mode at $-6$,
despite detecting it.
All of the parallel tempering algorithms produced samples that follow the
target distribution closely.
The linear temperature ladder with $\lambda = 2$ did well with the mode
at $-6$, while that with $\lambda = 1$ did very well with the mode at $8$.

    \begin{figure}[h]
    \begin{subfigure}[b]{0.3\textwidth}
        \includegraphics[width = \textwidth]{res/demo_01/density_00.png}
        \caption{Metropolis}
    \end{subfigure}
    \begin{subfigure}[b]{0.3\textwidth}
        \includegraphics[width = \textwidth]{res/demo_01/density_01.png}
        \caption{PT -- Linear $(1)$}
    \end{subfigure}
    \begin{subfigure}[b]{0.3\textwidth}
        \includegraphics[width = \textwidth]{res/demo_01/density_02.png}
        \caption{PT -- Linear $(2)$}
    \end{subfigure}
    \begin{subfigure}[b]{0.3\textwidth}
        \includegraphics[width = \textwidth]{res/demo_01/density_03.png}
        \caption{PT -- Geometric $(1.5)$}
    \end{subfigure}
    \begin{subfigure}[b]{0.3\textwidth}
        \includegraphics[width = \textwidth]{res/demo_01/density_04.png}
        \caption{PT -- Geometric $(2)$}
    \end{subfigure}
    \caption{
    Density plots for the gentle normal mixture example.
    The true density is shown dotted.
    }
    \label{fig:01_density}
    \end{figure}

The third example was motivated by the results of the second.
The target distribution is an equal mixture of normal distributions with means
$-20$, $-10$, $0$, and $10$, all with common variance $0.25$.
Only two of the four temperature ladders previously seen were used,
but each with $m = 5$ (a short ladder) and $m = 10$ (a long ladder),
in order to examine the effect of the ladder length.
The linear $\lambda = 1$ ladder was chosen based on its performance in the
previous example, while the geometric $\lambda = 1.5$ ladder was chosen for
the sake of comparison.
In this example, the variance of the proposal distribution had to be decreased
to 1 in order to maintain good acceptance rates.
    
    \begin{table}[h]
    \begin{tabular}{lrrrrr}
    \toprule
    & & \multicolumn{4}{c}{Parallel Tempering} \\
    \cmidrule(l){3-6}
    & Metropolis & Linear $(1S)$ & Linear $(1L)$ 
    & Geometric $(1.5S)$ & Geometric $(1.5L)$ \\
    \midrule
    \input{res/demo_02/diagnostics.tex}
    \bottomrule
    \end{tabular}
    \caption{
    Diagnostics for the difficult normal mixture example.
    }
    \label{tab:02_diag}
    \end{table}

The diagnostics in table~\ref{tab:02_diag} suggest that parallel tempering with
longer temperature ladders may actually be worse than with shorter ones, since
they lead to higher autocorrelations, lower effective sample sizes, and lower 
Geweke $p$-values.
However, the density plots in figure~\ref{fig:02_density} tell a very different
story. 
Only the chain using the long geometric temperature ladder managed to find
all of the modes.
Every other chain became trapped in the region around $-10$.

    \begin{figure}[h]
    \begin{subfigure}[b]{0.3\textwidth}
        \includegraphics[width = \textwidth]{res/demo_02/density_00.png}
        \caption{Metropolis}
    \end{subfigure}
    \begin{subfigure}[b]{0.3\textwidth}
        \includegraphics[width = \textwidth]{res/demo_02/density_01.png}
        \caption{PT -- Linear $(1S)$}
    \end{subfigure}
    \begin{subfigure}[b]{0.3\textwidth}
        \includegraphics[width = \textwidth]{res/demo_02/density_04.png}
        \caption{PT -- Geometric $(1.5L)$}
    \end{subfigure}
    \caption{
    Density plots for the difficult normal mixture example.
    Plots for the other parallel tempering samples were similar to that
    for the linear $(1S)$ case.
    The true density is shown dotted.
    }
    \label{fig:02_density}.
    \end{figure}

The (modest) success of the parallel tempering algorithm using a long geometric
temperature ladder with $\lambda = 1.5$ in the third example could be due to
the length of the ladder, but it could also be due to the magnitude of the
temperatures on the ladder.
In particular, $1.5^{10}$ is much larger than the maximum temperature of most of
the other ladders that have been applied.
Another question that arises is how quickly the parallel tempering algorithm
would arrive at the target distribution using that ladder.
In order to examine these points, a final example was run.
For this example, the number of iterations was increased to $250000$, and
a more diverse assembly of temperature ladders was used.
The Metropolis algorithm was not even considered, due to its poor performance
in the previous examples.

    \begin{table}[h]
    \begin{tabular}{lrrrrr}
    \toprule
    & \multicolumn{5}{c}{Parallel Tempering} \\
    \cmidrule(l){2-6}
    & Linear $(5S)$ & Geometric $(1.1L)$ & Geometric $(1.5L)$ 
    & Geometric $(2S)$ & Geometric $(3S)$ \\
    \midrule
    \input{res/demo_03/diagnostics.tex}
    \bottomrule
    \end{tabular}
    \caption{
    Diagnostics for the second attempt at the difficult normal mixture example.
    }
    \label{tab:03_diag}
    \end{table}

Diagnostics for this example are presented in table~\ref{tab:03_diag}.
It first appears that the long geometric ladder with $\lambda = 1.1$ did
exceptionally well, while either the linear ladder or the short geometric
ladder with $\lambda = 2$ comes in second.
However, the density plots, in figure~\ref{fig:03_density}, show that the
long geometric ladder with $\lambda = 1.1$ is the only case where the
algorithm still became trapped in the region around $-10$.
Furthermore, the results seem to indicate that a longer temperature ladder
is not necessarily better for exploring the state space.
Instead, having extreme temperatures on the ladder improves transit around
the state space.

    \begin{figure}[h]
    \begin{subfigure}[b]{0.3\textwidth}
        \includegraphics[width = \textwidth]{res/demo_03/density_00.png}
        \caption{PT -- Linear $(5S)$}
    \end{subfigure}
    \begin{subfigure}[b]{0.3\textwidth}
        \includegraphics[width = \textwidth]{res/demo_03/density_01.png}
        \caption{PT -- Geometric $(1.1L)$}
    \end{subfigure}
    \begin{subfigure}[b]{0.3\textwidth}
        \includegraphics[width = \textwidth]{res/demo_03/density_02.png}
        \caption{PT -- Geometric $(1.5L)$}
    \end{subfigure}
    \begin{subfigure}[b]{0.3\textwidth}
        \includegraphics[width = \textwidth]{res/demo_03/density_03.png}
        \caption{PT -- Geometric $(2S)$}
    \end{subfigure}
    \begin{subfigure}[b]{0.3\textwidth}
        \includegraphics[width = \textwidth]{res/demo_03/density_04.png}
        \caption{PT -- Geometric $(3S)$}
    \end{subfigure}
    \caption{
    Density plots for the second attempt at the difficult normal mixture 
    example.
    The true density is shown dotted.
    }
    \label{fig:03_density}.
    \end{figure}

\FloatBlock
\chapter*{Conclusions}
There is no question that the parallel tempering algorithm outperforms the
Metropolis (and by extension, Metropolis-Hastings) algorithm, not only with
respect to exploration of the state space, but also with respect to the rate
of convergence.
Because its implementation requires only a few simple modifications of the
regular Metropolis algorithm, there is little reason not to adopt parallel
tempering as the default method of choice whenever a better, more specialized
method does not present itself.
A potential cost is in compute time, because running multiple Markov chains is
considerably more expensive than running one.
However, this is somewhat mitigated by the rate of convergence; a short run of
parallel tempering may be equivalent to a much longer run of
Metropolis-Hastings.
Moreover, this method is amenable to parallelization, albeit with the caveat
that threads must periodically synchronize in order to swap states.

Despite the considerations of this paper, the best choice of temperature ladder
remains murky, and may be highly situational. 
However, including extreme temperatures and using a long ladder, so that states
from those extremes successfully trickle down, seems pertinent based on some
of the evidence presented.
Furthermore, it is intuitive: high temperature replicas are very nearly flat,
so the state space can be explored rapidly, and adjacent replicas have similar
temperatures in a long chain, so state swaps are frequently accepted.
This issue should, nonetheless, be examined in greater detail.

The frequency with which state swaps should occur was not investigated,
but this is an important point. 
While decreasing the frequency seems like it would only cause a corresponding
dampening of the benefits of parallel tempering for the target chain,
the frequency could be increased by requiring more than one pair of chains
to swap simultaneously.

Another point which was neglected here is the role of the proposal distribution
in the effectiveness of the parallel tempering algorithm.
High-variance proposals or heavy-tailed proposals (e.g.\ Laplace) may
provide even faster exploration of the state space if used to update the high
temperature replicas, while lower variance proposals could still be used for
the target and lower temperature replicas.
This approach may or may not disrupt the theoretical properties of the chain,
which is a topic that goes well beyond the details presented.

\chapter*{Extensions}
Many extensions are possible to the provided code.
It would be fairly easy to add support for multivariate sampling,
and to add user control of the swap frequency.
Slightly more difficult would be tracking the acceptance rates or even the
entire histories for all of the replicas, but these would be very useful to
have for analysis and diagnosis of performance.

A more novel, but still quite useful, thing would be to implement the algorithm
in parallel on a multiprocessor, or even a GPU.
The latter may require significant tuning of the algorithm to make it
performant, possibly to the extent of redesigning parts of it.
One approach is to use an extremely long temperature ladder
(64 to 256 replicas), running each replica on its own thread.
Issues that might arise include the cost of communication between threads,
and the ability of the threads to keep the GPU running at full load.

Finally, a more theoretical approach may be warranted here.
It's possible that useful results about the workings of parallel tempering
are within reach of a diligent theoretician. 

\clearpage
\chapter*{Source Code}
\hrule
\lstinputlisting{../mcmc_algorithms.py}
\hrule
\lstinputlisting{../mcmc_demos.py}
\hrule
\lstinputlisting{../postproc.R}
\end{document}

